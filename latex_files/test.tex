\documentclass{article}
\usepackage[utf8]{inputenc}
\usepackage{amsmath}
\setlength{\jot}{10pt}
\setlength\parindent{0pt}
\usepackage{listings}
\usepackage{graphicx}
\usepackage{xcolor}
\usepackage{titlesec}
\usepackage{adjustbox}
\usepackage{tikz}
\usepackage{tzplot}
\usepackage{float}
\usepackage{amssymb}
\usepackage{wasysym}
\title{Macroeconomics 3 Problem Set 1}
\author{Jinxu Mi, Norbert Monti, Justine Nayral, \\ Peter Tkatchenko, Yuxuan Xu}
\date{September 2023}

\begin{document}

\maketitle

\section{Exercise 1}
The economy is assumed to be deterministic. The consumer wants to maximise their discounted utility by choosing each period the optimal level of consumption and amount of their future wealth to consume in the future $\{(c_t, W_{t+1})\}^{+\infty}_{t=0}$. In this economy, it is assumed agents cannot borrow and $W_t \geq 0$, $\forall t$. \\

The consumer's maximization problem is as follows:

\begin{gather*}
    \max_{\{(c_t, W_{t+1})\}^{+\infty}_{t=0}} \sum^\infty_{t=0} \beta^tu(c_t)
\end{gather*}

subject to the budget constraint

\begin{gather*}
    c_t + W_{t+1} \leq RW_t \quad \forall t \quad  \mbox{with} \quad W_0 \quad \mbox{given}.
\end{gather*}

We assume the budget constraint holds with equality, implied by the monotonicity of the utility function (strictly increasing in consumption). \\
%As the utility is increasing with $c_t$, the household has interest to spend all his budget by either consuming at $t$ or equivalently delay his consumption at $t+1$ by saving for next period a part of his wealth $W_{t+1}$. \\

The Lagrangian of the problem can be written as:

\begin{gather*}
    \mathcal{L} = \sum^\infty_{t=0} \beta^tu(c_t) + \lambda_t\sum^\infty_{t=0}[RW_t - c_t - W_{t+1}].
\end{gather*}

The first-order conditions allow us to find the following results:

\begin{gather*}
\frac{\partial \mathcal{L}}{\partial c_t} = 0 \implies \beta^tu'(c_t) = \lambda_t \\
\frac{\partial \mathcal{L}}{\partial c_{t+1}} = 0 \implies \beta^{t+1}u'(c_{t+1}) = \lambda_{t+1} \\
\frac{\partial \mathcal{L}}{\partial W_{t+1}} = 0 \implies -\lambda_t + \lambda_{t+1}R = 0 \\
\implies \frac{\lambda_t}{\lambda_{t+1}} = R
\end{gather*}

The two expressions found for $\frac{\lambda_t}{\lambda_{t+1}}$ give the Euler's equation:

\begin{gather*}
        \frac{\lambda_t}{\lambda_{t+1}} = \frac{u'(c_t)}{\beta u'(c_{t+1})}
        \implies R = \frac{u'(c_t)}{\beta u'(c_{t+1})} \\
        u'(c_{t+1})\beta R = u'(c_t)
\end{gather*}

The Euler's equation describes the optimal trade-off between consumption and saving. Indeed, saving $\epsilon$ today reduces utility by $u'(c_t)\epsilon$ but, using this saving for consumption tomorrow increases tomorrow's utility by $\beta Ru'(c_{t+1})\epsilon$. As such, at the optimum, the marginal cost of saving should be equal to the marginal benefit of saving.

\subsection{a}

If $\beta R = 1$ then:

\begin{gather*}
        \beta R = \frac{u'(c_t)}{u'(c_{t+1})} \\
        1 = \frac{u'(c_t)}{u'(c_{t+1})} \\
        \Leftrightarrow u'(c_{t+1}) = u'(c_t) \\
        \Leftrightarrow c_{t+1} =c_{t} \quad \mbox{assuming the utility function is strictly concave}
\end{gather*}

This implies that the marginal utility of consumption at time $t$ is equal to the marginal utility of consumption at time $t+1$.
As such, $c_t=c_{t+1}$, consumption today is equal to consumption tomorrow. Consumption is perfectly smooth over time for a strictly concave utility function. Strict concavity describes the taste for smooth consumption.

\subsection{b}

If $\beta R > 1$ then:

\begin{gather*}
        \beta R = \frac{u'(c_t)}{u'(c_{t+1})} \\
        1 > \frac{u'(c_t)}{u'(c_{t+1})} \\
        \Leftrightarrow u'(c_{t+1}) > u'(c_t)
\end{gather*}

This implies that the marginal utility of consumption at time $t$ is less than the marginal utility of consumption at time $t+1$. If we assume a strictly concave utility function, such that the individual prefers a smooth consumption stream, $u'(c_t)$ is strictly decreasing in $c_t$.

\begin{gather*}
    u'(c_{t+1}) > u'(c_t) \\
    \Leftrightarrow c_{t+1} < c_t
\end{gather*}

As such, if $\beta R>1$, 
consumption today will be higher than consumption tomorrow. Indeed, the marginal gains $\beta R$ of saving an amount $\epsilon$ today and reducing current consumption by the same amount will be lower. As such, the household has an interest in consuming more today rather than waiting for tomorrow. The effect of a higher return on saving is ambiguous depending on whether the income or the substitution effect dominates. \\

If $\beta R < 1$ then:

\begin{gather*}
        \beta R = \frac{u'(c_t)}{u'(c_{t+1})} \\
        1 < \frac{u'(c_t)}{u'(c_{t+1})} \\
        \Leftrightarrow u'(c_{t+1}) < u'(c_t)
\end{gather*}

This implies that the marginal utility of consumption at time $t$ is greater than the marginal utility of consumption at time $t+1$. If we assume a strictly concave utility function, such that the individual prefers a smooth consumption stream, $u'(c_t)$ is strictly decreasing in $c_t$.

\begin{gather*}
    u'(c_{t+1}) < u'(c_t) \\
    \Leftrightarrow c_{t+1} > c_t
\end{gather*}

As such, if $\beta R<1$, consumption today will be lower than consumption tomorrow. Indeed, the marginal gains $\beta R$ of saving an amount $\epsilon$ today and reducing current consumption by the same amount will be higher. As such, the household has interest to consume less today and save more to consume tomorrow. It can be due to a more patient household (high $\beta$). The effect of a lower return on saving is ambiguous depending on whether the income or the substitution effect dominates.

\section{Exercise 2}

\subsection{a}

An Arrow-Debreu Equilibrium is a market equilibrium with agents' allocation and a price system. Each agent optimizes their lifetime utility subject to the budget constraint, resource constraints, and transversality conditions,  while the aggregate resource feasibility constraint is met (i.e. markets clear). They choose the optimal consumption plan $\{c_t \}^{\infty}_{t=0}$. \\

We define an ADE for this problem:\\

For each household $i$, we have an allocation $\left\{ c^{i}_{t}\right\}^{\infty }_{t=0}  $, a price system $\left\{ p_{t}\right\}^{\infty }_{t=0}  $ such that given prices $\left\{ p_{t}\right\}^{\infty }_{t=0}  $, $\left\{ c^{i}_{t}\right\}^{\infty }_{t=0}  $ solves for any agent $i$,

\begin{gather*}
    \max_{\{c^{i}_{t}\}^{\infty}_{t=0}} \sum^{\infty }_{t=0} \beta^{t} u\left( c^{i}_{t}\right) \forall i\\
    \mbox{with}\quad u(c_t^i)=\ln c_t^i
\end{gather*}

subject to the budget constraint:

\begin{gather*}
    \sum^{\infty }_{t=0} p_{t}e^{i}_{t} \geq \sum^{\infty }_{t=0} p_{t}c^{i}_{t}.
\end{gather*}

Finally, we have the market clearing condition for non-storable goods. In each period, the total quantity consumed should be equal to the total endowment: 

\begin{gather*}
    \sum_{i\in I} e^{i}_{t}=\sum_{i\in I} c^{i}_{t} \quad \forall t\geq 0.
\end{gather*}

We assume an endowment economy with no production function. \\

\subsection{b}

\begin{gather*}
    L=\sum^{+\infty }_{t=0} \beta^{t} \ln\  c^{i}_{t}+\lambda^i \left( \sum^{\infty}_{t=0}p_{t}e^{i}_{t}-\sum^{\infty}_{t=0}p_{t}c^{i}_{t}\right)  
\end{gather*}

In the Arrow-Debreu setup, we have only one Lagrangian multiplier as the agents are taking decisions in period 0 and they have perfect information about prices and the state of the world (the endowment process).\\


We solve for the FOC, $\forall i \in I$:

\begin{gather*}
\frac{\partial L}{\partial c^{i}_{t}} =\beta^{t} \frac{1}{c^{i}_{t}} -p_{t}\lambda^i =0\Leftrightarrow \frac{\beta^{t} }{c^{i}_{t}} =\lambda^i p_{t} \\
\frac{\partial L}{\partial c^{i}_{t+1}} =\beta^{t+1} \frac{1}{c^{i}_{t+1}} -p_{t+1}\lambda^i =0 \Leftrightarrow \frac{\beta^{t+1} }{c^{i}_{t+1}} =\lambda^i p_{t+1}
\end{gather*}

We can compute the Euler Equation:

\begin{gather*}
\frac{p_{t+1}\lambda^i }{p_{t}\lambda^i } =\frac{\beta c^{i}_{t}}{c^{i}_{t+1}}
\end{gather*}

\begin{gather*}
    \beta \frac{c^{i}_{t}}{c^{i}_{t+1}}=\frac{p_{t+1}}{p_{t}} 
\end{gather*}

%If not the case then:

%\begin{gather*}
%\  c^{i}_{t}\neq c^{i}_{t+1}\  \  \Longrightarrow \sum^{}_{i\in I} c^{i}_{t}\neq \sum^{}_{i\in I} c^{i}_{t+1}
%\end{gather*}

%which violates the market-clear condition. We can conclude agents have an incentive to smooth their consumption as $c^{i}_{t}=c^{i}_{0}$. \\

For $i=\{1,2\}$, we have 2 Euler's equations:

\begin{gather*}
    \beta \frac{c^{1}_{t}}{c^{1}_{t+1}}=\frac{p_{t+1}}{p_{t}} \quad \mbox{and}\quad \beta \frac{c^{2}_{t}}{c^{2}_{t+1}}=\frac{p_{t+1}}{p_{t}}
\end{gather*}

The two agents face the same equilibrium prices. As such:
\begin{gather*}
    \frac{c^{1}_{t}}{c^{1}_{t+1}}=\frac{c^{2}_{t}}{c^{2}_{t+1}}
\end{gather*}

By the market clearing equation, and because the endowment is the same for all periods studied:

\begin{gather*}
    c_t^1+c_t^2=e_t^1+e_t^2=8 \quad \forall t \geq 0.
\end{gather*}

From the Euler's equation:
\begin{gather*}
    c_{t+1}^1=c_t^1 \times \frac{c_{t+1}^2}{c_{t}^2}. \\
\end{gather*}

From the market clearing equation:

\begin{gather*}
    c_{t+1}^2+c_{t+1}^1=c_t^1 \times \frac{c_{t+1}^2}{c_{t}^2}+ c_{t+1}^2=8 \\
    \Leftrightarrow c_{t+1}^2\times \left(\frac{c_t^1}{c_t^2}+1\right)=8 \\
    \Leftrightarrow c_{t+1}^2\times \left(\frac{c_t^1+c_t^2}{c_t^2}\right)=8\\
    \Leftrightarrow c_{t+1}^2\times \left(\frac{8}{c_t^2}\right)=8 \quad\mbox{because, by market clearing}\quad c_t^1+c_t^2=8\\
    \Leftrightarrow c_{t+1}^2=c_{t}^2 \quad\forall t.
\end{gather*}

Applying the same reasoning for agent 1: $c_{t+1}^1=c_{t}^1$\\

As such, both agents trade in order to perfectly smooth their consumption over time.\\

From the Euler's equation: $\beta \frac{c^{i}_{t}}{c^{i}_{t+1}}=\frac{p_{t+1}}{p_{t}}$.\\
Because $c_t^{i}=c_{t+1}^{i}=c_0^{i}\quad\forall t, i$:

\begin{gather*}
    \beta \frac{c^{i}_{0}}{c^{i}_{0}}=\frac{p_{t+1}}{p_{t}}
    \Leftrightarrow \frac{p_{t+1}}{p_{t}}=\beta\\
    \Leftrightarrow p_{t+1}=\beta p_{t}.
\end{gather*}

By normalizing $p_{0}=1$ (numéraire) we have the price system:

\begin{gather*}
p_{0}=1 \ p_{1}=\beta \ p_{2}=\beta^{2}\quad ...\quad p_{t}=\beta ^{t}
\end{gather*}

The budget constraint thus implies:

\begin{gather*}
    c^{i}_{0}\sum^{\infty }_{t=0} p_{t}=\sum^{\infty }_{t=0} p_{t}e^{i}_{t}\\
    c^{i}_{0}\left( 1+\beta +\beta^{2} +\beta^{3} +...\right)  =\sum^{\infty }_{t=0} p_{t}e^{i}_{t}\\
    c^{i}_{0}.\frac{1}{1-\beta}  =\sum^{\infty }_{t=0} p_{t}e^{i}_{t}\\
c^{i}_{0}=\left( 1-\beta \right)  \sum^{\infty }_{t=0} p_{t}e^{i}_{t}.
\end{gather*}

For the first agent with $e^{1}=(5,3,5,3, \hdots )$:
\[
c^{1}_{0}=\left( 1-\beta \right)  \left( 5\times 1+3\times \beta +5\times \beta^{2} +3\times \beta^{3} +...\right)  =\left( 1-\beta \right)  (\frac{5}{1-\beta^{2} } +\frac{3\beta }{1-\beta^{2} } )=\frac{5+3\beta }{1+\beta } 
\]
For the second agent with $e^{2}=(3,5,3,5, \hdots)$:
\[
c^{2}_{0}=\left( 1-\beta \right)  \left( 3\times 1+5\times \beta +3\times \beta^{2} +5\times \beta^{3} +...\right)  =\left( 1-\beta \right)  (\frac{3}{1-\beta^{2} } +\frac{5\beta }{1-\beta^{2} } )=\frac{3+5\beta }{1+\beta } 
\]
These are the allocations of the 2 agents.\\

As such, the Arrow-Debreu, competitive equilibrium is :
\begin{gather*}
    p_t=\beta^t\quad\mbox{with}\quad p_0=1\\
    c_t^1=\frac{5+3\beta}{1+\beta} \quad\mbox\quad c_t^2=\frac{3+5\beta}{1+\beta}\quad\forall t \geq 0
\end{gather*}

\subsection{c}

In a sequential market equilibrium, agents choose every period their optimal level of consumption and assets $\{(c_t, a_{t+1}\}^{+\infty}_{t=0}$. The one period interest rate is $r_t$. The housholds maximise their utility subject to their budget constraint at each period $t$ and the feasibility condition is met (i.e. the markets clear).\\

$\forall i=\{1,2\}$, the maximization problem is the following:\\

\begin{gather*}
    \max_{\{c_t^i, a_{t+1}^i\}^{+\infty}_{t=0}}\sum_{t=0}^{\infty}\beta^tu(c_t^i)\quad\mbox{with}\quad u(c_t^i)=\ln c_t^i\quad\forall t \geq 0
\end{gather*}

subject to the budget constraint:

\begin{gather*}
    c_t^i+a^i_{t+1}\leq e_t^i+(1+r_t)a_t^i
\end{gather*}

Because the utility function is increasing with consumption, the household has interest to consume or to save for future consumption all his endowment every period.\\

We impose a no-Ponzi scheme. It is a lower bound to the amount the agent can borrow from the other agent. Otherwise, the amount borrowed could go to $-\infty$ and consumption would be infinite. Each agent should be able to repay his debt.

\begin{gather*}
    a_{t+1}\leq-\Bar{A}\quad\mbox{with}\quad \bar{A}\in\mathbf{R}_+\quad\mbox{with}\quad a_0^i=0\quad\forall i\in\{1,2\}
\end{gather*}

The feasibility constraint is met. The good is non storable, as such the total amount consumed is equal to the total endowment in every period. Moreover, the total assets are 0 in each period (one agent is the creditor and the other one is the borrower).\\

\begin{gather*}
    \sum_{i\in I}c_t^i=\sum_{i\in I}e_t^i\quad\forall t\geq0\quad\mbox{good market clears}\\
    \sum_{i\in I}a_t^i=0\quad\forall t\geq0\quad\mbox{asset market clears}
\end{gather*}

Finally, the consumer do not overaccumulate capital
\begin{gather*}
    \lim_{t\rightarrow +\infty} p_ta_{t+1}^i \geq 0
\end{gather*}

\subsection{d}
We set the Lagrangian, $\forall i\in I$:

\begin{gather*}
    \mathcal{L}=\sum_{t=0}^{+\infty}\beta^t\ln{c_t^i}-\sum_{t=0}^{+\infty}\lambda_t^i[c_t^i+a_{t+1}^1-e_t^1-(1+r_t)a_t^i]
\end{gather*}

The FOCs are the following, $\forall i\in I$:

\begin{gather*}
    \frac{\partial\mathcal{L}}{\partial c_t^i}=0\Leftrightarrow\lambda_t^i=\frac{\beta^t}{c_t^i}\\
    \frac{\partial\mathcal{L}}{\partial a_{t+1}^i}=0\Leftrightarrow\lambda_t^i=\beta(1+r_{t+1})\lambda_{t+1}^i\\
    \frac{\partial\mathcal{L}}{\partial \lambda_{t}^i}=0\Leftrightarrow c_t^i+a_{t+1}^i=e_t^i+(1+r_t)a_t^i
\end{gather*}

As such:

\begin{gather*}
    \frac{\beta^t}{c_t^i}=\frac{\beta^{t+1}}{c_{t+1}^i}(1+r_{t+1})\\
    \Leftrightarrow{c_{t+1}^i}=\beta(1+r_{t+1}){c_{t}^i}
\end{gather*}

Sum across all agents and we will have:

\begin{gather*}
    \sum_{i\in I}c_{t+1}^i=\sum_{i\in I}\beta(1+r_{t+1}){c_{t}^i}\\
    \sum_{i\in I}c_{t+1}^i=\beta(1+r_{t+1})\sum_{i\in I}{c_{t}^i}
\end{gather*}

From the feasibility constraint and because the total endowment is the same for each period:

\begin{gather*}
    \sum_{i\in I}c_t^i= \sum_{i\in I}c_{t+1}^i=\sum_{i\in I}e_t^i=8\\
\end{gather*}

Which implies that:

\begin{gather*}
    \beta=\frac{1}{1+r_{t+1}}\\
    \Leftrightarrow 1=\beta\left(1+r_{t+1}\right)=\beta\left({1+r}\right)
\end{gather*}

As such, $r_t=r\quad\forall t\geq0$, and by the Euler equation:

\begin{gather*}
    c_{t+1}^i=c_t^i=c^i\\
    \beta=\frac{1}{1+r}
\end{gather*}

The consumption is perfectly smooth over time.\\

Now, we can try to find the prices defined in the Arrow-Debreu equilibrium such that the allocation in the sequential market is equivalent to the one in the competitive equilibrium.\\

The household face a sequence of budget constraints for each periods of time, and assuming $a_0^i=0$:

\begin{gather*}
    c^i+a_1^i=e_0^i\\
    c^i+a_2^i=e_1^i+(1+r)a_1^i\\
    c^i+a_3^i=e_2^i+(1+r)a_2^i
\end{gather*}
Using this three equations:
\begin{gather*}
    c^i+a_2^1=e^i_1+(1+r)(e_0^i-c^i)\\
    \Leftrightarrow c^i+(1+r)c^i+a_2^i=e^i_1+e_0^i\\
    c^i+\frac{c^i}{1+r}+\frac{a_2^i}{1+r}=e_0^i+\frac{e^i_1}{1+r}\\
    c^i+\frac{c^i}{1+r}+\frac{c^i}{(1+r)^2}+\frac{a_3^i}{1+r}=e_0^i+\frac{e^i_1}{1+r}+\frac{e^i_2}{(1+r)^2}
\end{gather*}
We can continue the same logic until an arbitrary period T:
\begin{gather*}
    \sum^T_{t=0}\frac{c^i}{\prod^t_{j=0}(1+r)}+\frac{a^i_{T+1}}{\prod^T_{j=0}(1+r)}= \sum^T_{t=0}\frac{e^i_t}{\prod^t_{j=0}(1+r)}
\end{gather*}
At the limit:
\begin{gather*}
    \lim_{T\rightarrow+\infty}\sum^T_{t=0}\frac{c^i}{\prod^t_{j=0}(1+r)}+\lim_{T\rightarrow+\infty}\frac{a^i_{T+1}}{\prod^T_{j=0}(1+r)}= \lim_{T\rightarrow+\infty}\sum^T_{t=0}\frac{e^i_t}{\prod^t_{j=0}(1+r)}
\end{gather*}

The transversality condition, as well as the fact that $\frac{1}{1+r} < 1$ implies that:

\begin{gather*}
    \lim_{T\rightarrow+\infty}\frac{a^i_{T+1}}{\prod^T_{j=0}(1+r)}=0
\end{gather*}

Therefore:

\begin{gather}
    \lim_{T\rightarrow+\infty}\sum^T_{t=0}\frac{c^i}{\prod^t_{j=0}(1+r)} = \lim_{T\rightarrow+\infty}\sum^T_{t=0}\frac{e^i_t}{\prod^t_{j=0}(1+r)}
\end{gather}

We want to find the price $\{p_t\}^{\infty}_{t=0}$ such that we have the following structure in the Arrow-Debreu equilibrium:

\begin{gather}
    \sum^{+\infty}_{t=0}p_tc^i=\sum^{+\infty}_{t=0}p_te^i_t
\end{gather}

For (1) and (2) to be equal, we need to set the prices such that $p_t=\frac{1}{\prod^t_{j=0}(1+r)}$.

We proved previously that $\beta=\frac{1}{1+r}$. As such:

\begin{gather*}
    p_t=\frac{1}{\prod^t_{j=0}(1+r)}\\
    p_t=\prod^t_{j=0}\frac{1}{(1+r)}\\
    p_t=\left(\frac{1}{(1+r)}\right)^t\\
    p_t=\beta^t
\end{gather*}

which is the same price as the in the Arrow-Debreu equilibrium.\\

Using this price and the budget constraint, we can find the optimal consumption bundles for each agents.\\

For consumer 1: $e^1=(5, 3, 5,3, \hdots)$\\

\begin{gather*}
    \sum^{\infty}_{t=0}c^1_tp_t=\sum^{\infty}_{t=0}p_te_t^1\\
    c^1\sum^{\infty}_{t=0}\beta^t=\beta^0 \times 5+\beta^1 \times 3+\beta^2 \times 5+\beta^3 \times 3 \hdots\\
    \frac{c^1}{1-\beta}=\frac{5}{1-\beta^2}+\beta  \times \frac{3}{1-\beta^2}\\
    c^1=\frac{5+3\beta}{1+\beta}
\end{gather*}

For consumer 2: $e^2=(3, 5,3,5, \hdots)$\\

\begin{gather*}
    \sum^{\infty}_{t=0}c^2_tp_t=\sum^{\infty}_{t=0}p_te_t^2\\
    c^2\sum^{\infty}_{t=0}\beta^t=\beta^0 \times 3+\beta^1 \times 5+\beta^2 \times 3+\beta^3 \times 5  \times \hdots\\
    \frac{c^2}{1-\beta}=\frac{3}{1-\beta^2}+\beta\frac{5}{1-\beta^2}\\
    c^2=\frac{5\beta+3}{1+\beta}
\end{gather*}

For each period, we get:

\begin{gather*}
    c^1+c^2=\frac{5\beta+3}{1+\beta}+\frac{3\beta+5}{1+\beta}=8
\end{gather*}

which is the initial endowment, as expected from the market clearing equation.\\

We can now recover how the assets are going to be in each period.\\

From the budget constraint, for $t=1$, we have for consumer 1:

\begin{gather*}
    a_1^1=e_1^1-c^1=5-\frac{5+3\beta}{1+\beta}=\frac{2\beta}{1+\beta}
\end{gather*}

And for consumer 2:

\begin{gather*}
    a_1^2=e_1^2-c^2=3-\frac{3+5\beta}{1+\beta}=-\frac{2\beta}{1+\beta}
\end{gather*}


In period 1, agent 2 is rich (endowment of 5)  is saving, since their asset position is positive and agent 1, the poor agent (endowment of 3) is borrowing since their asset position is negative. In other words, agent 2 is lending to agent 1. We can continue this process and discover how the assets are going to evolve.\\

Going one period further:\\

For consumer 1:
\begin{gather*}
    a^1_2=e_1^1-c_1^1+a^1_1(1+r)=5-\frac{5+3\beta}{1+\beta}+(1+r)\times\frac{2\beta}{1+\beta}=\\
    a^1_2=5-\frac{5+3\beta}{1+\beta}+\frac{1}{\beta}\times\frac{2\beta}{1+\beta}\\
    a^1_2=\frac{4}{1+\beta}
\end{gather*}

By symmetry for consumer 2:

\begin{gather*}
    a^2_2=-\frac{4}{1+\beta}
\end{gather*}

And so on, and so forth.\\

Finally, the sequential market equilibrium is given by:

\begin{gather*}
    \beta=\frac{1}{1+r}\\
    p_t=\beta^t\\
    c^1=\frac{3\beta+5}{1+\beta}\quad\forall t \\
    c^2=\frac{5\beta+3}{1+\beta}\quad\forall t \\
    a_1^1, a_1^2, a_2^1, a_2^2, ...
\end{gather*}

\subsection{e}

Now, the total endowment of this economy is not constant over time. It can be seen as periods of booms (high endowment $Y_H$ equal to 9) and recession (low endowment $Y_L$ equal to 7).

\begin{gather*}
    Y_t=e_t^1+e_t^2\quad\forall t\leq 0
\end{gather*}

such that: $Y=(Y_H, Y_L, Y_H, Y_L, ...)=(9,7,9,7,...)$, $Y_L=7$ and $Y_H=9$. \\

What we expect in this economy is that consumption is smooth however, because the good is not storable, agents will consume the same bundle every time the total endowment is high (H) and another bundle when the economy is low (L). We expect a higher price when the economy is in L compared to when the economy is in H since the consumption good will be more valuable (scarcer) when the economy is in recession compared to the case when the economy is in boom.\\

Household's problem:

\begin{gather*}
  \max_{\{c_t^i\}^{+\infty}_{t=0}}u(c_t^i)=\sum_{t=0}^{+\infty}\beta^t\ln c_t^i
\end{gather*}

subject to:

\begin{gather*}
    \sum^{\infty }_{t=0} p_{t}e^{i}_{t} \geq \sum^{\infty }_{t=0} p_{t}c^{i}_{t}.
\end{gather*}

We set the Lagrangian to the problem, $\forall i\in I$:

\begin{gather*}
    \mathcal{L}=\sum_{t=0}^{+\infty}\beta^t\ln c_t^i+\lambda^i\left(\sum_{t=0}^{+\infty}p_t(e_t^i-c_t^i)\right)
\end{gather*}

In the Arrow-Debreu structure, we have only one Lagrangian multiplier as the agents are taking decisions in period 0 and they have information about the prices in all periods.\\

The FOCs are the following:

\begin{gather*}
    \frac{\partial\mathcal{L}}{\partial c_t^i}=0\Leftrightarrow \frac{\beta^t}{c_t^i}=\lambda^ip_t\quad\forall t\geq 0
\end{gather*}

The equation above is valid for all $t$ including $t=0$, which has a result:

\begin{gather*}
    \frac{\beta^0}{c_0^i}=\lambda^ip_0
\end{gather*}

We set $p_0=1$, as the numéraire, the price according to which all other prices are relative.\\

Using the two previous equations, we get:

\begin{gather*}
    p_t=\frac{\beta^tc_0^i}{c_t^i}\quad\forall t\leq0\quad\forall i\in I
\end{gather*}

Since we are in competitive equilibrium, both agents $i\in\{1,2\}$ face the the same price:

\begin{gather*}
    p_t =\frac{\beta^t c_0^1}{c_t^1}=\frac{\beta^t c_0^2}{c_t^2}\quad\forall t\leq0
\end{gather*}

Assume, in period t, we are in the period of large endowment with $c_t^1=c^1_H$ and $c_t^2=c^2_H$. The next period is a period of low endowment: $c_{t+1}^1=c^1_L$ and $c_{t+1}^2=c^2_L$.\\

Using the previous equation, we get:

\begin{gather*}
    \frac{c_H^1}{c_L^1}=\frac{c_H^2}{c_L^2}
\end{gather*}

Similarly, we get an equivalent equation if we do the inverse. Assume, in period t, we are in the period of low endowment with $c_t^1=c^1_L$ and $c_t^2=c^2_L$. The next period is a period of large endowment: $c_{t+1}^1=c^1_H$ and $c_{t+1}^2=c^2_H$.\\

Using the previous equation, we get:

\begin{gather*}
    \frac{c_L^1}{c_H^1}=\frac{c_L^2}{c_H^2}
\end{gather*}

For each period of low or large endowments, the market clear:

\begin{gather*}
    c_L^1+c_L^2=7\\
    c_H^1+c_H^2=9
\end{gather*}

We observe that:

\begin{gather*}
    \frac{c_H^1}{c_L^1}=\frac{9-c_H^1}{7-c_L^1}\Leftrightarrow c_H^1(7-c_L^1)=c_L^1(9-c_H^1)\\
    \Leftrightarrow 7c_H^1=9c_L^1\\
    \Leftrightarrow \frac{c_H^1}{c_L^1}=\frac{9}{7}
\end{gather*}

The agents consume the same bundle of good when the endowment is high: $c_0^1=c_H^1$, and face the price $p_H$. Similarly, the agents consume the same bundle of good when the endowment is low: $c_1^1=c_L^1$, and face the price $p_L$. The economy is high at $t=0$ and low at period 1.\\

A such, using the equation for the price we found before:

\begin{gather*}
    \forall t\geq0 \quad\forall i\in I\\
    p_t =\frac{\beta^t c_0^i}{c_t^i}=\frac{\beta^t c_H^i}{c_t^i}\\
    p_H =\frac{\beta^t c_H^i}{c_H^i}\Leftrightarrow p_H=\beta^t\\
    p_L=\frac{\beta^t c_H^i}{c_L^i}\Leftrightarrow p_L=\beta^t\times\frac{9}{7}
\end{gather*}

To find the allocation, we use the budget constraint:

\begin{gather*}
    \sum^{+\infty}_{t=0}p_tc_t^i=\sum^{+\infty}_{t=0}p_te_t^i \quad\forall i\in I
\end{gather*}

For consumer 1, this equation becomes:

\begin{gather*}
p_Hc_H^1+p_Lc_L^1+p_Hc_H^1+p_Lc_L^1+\hdots=p_He_H^1+p_Le_L^1+p_He_H^1+p_Le_L^1+\hdots\\
\end{gather*}

Using $p_L=\frac{9}{7}\beta^t$

\begin{gather*}
    \beta^0c_H^1+\frac{9}{7}\beta^1c_L^1+\beta^2c_H^1+\frac{9}{7}\beta^3c_L^1+\hdots=\beta^0e_H^1+\frac{9}{7}\beta^1e_L^1+\beta^2e_H^1+\frac{9}{7}\beta^3e_L^1+\hdots\\
\end{gather*}

Using $c_L^1=\frac{7}{9}c_H^1$

\begin{gather*}
    \beta^0c_H^1+\beta^1c_H^1+\beta^2c_H^1+\beta^3c_H^1+\hdots=\beta^0e_H^1+\frac{9}{7}\beta^1e_L^1+\beta^2e_H^1+\frac{9}{7}\beta^3e_L^1+\hdots\\
    \beta^0c_H^1+\beta^1c_H^1+\beta^2c_H^1+\beta^3c_H^1+\hdots=5\beta^0+3\times\frac{9}{7}\beta^1+5\times\beta^2+3\times\frac{9}{7}\beta^3+\hdots\\
    \frac{1}{1-\beta}c_H^1=5\frac{1}{1-\beta^2}+3\times\frac{9}{7}\frac{\beta}{1-\beta^2}\\
   c_H^1=\frac{5+\frac{27}{7}\beta}{1+\beta}
\end{gather*}

We deduce $c_L^1= \frac{7}{9}c_H^1$:

\begin{gather*}
    c_L^1=\frac{7}{9}\left(\frac{5+\frac{27}{7}\beta}{1+\beta}\right)\\
     c_L^1=\frac{35+27\beta}{9(1+\beta)}
\end{gather*}
Using the market clearing equation, we can recover $c_L^2$ and $c_H^2$:
\begin{gather*}
    c_H^2=e_H^1+e_H^2-c_H^1=9-\frac{5+\frac{27}{7}\beta}{1+\beta}=\frac{\frac{36}{7}+4}{1+\beta}\\
    c_L^2=e_L^1+e_L^2-c_L^1=7-\frac{35+27\beta}{9(1+\beta)}=\frac{28+36\beta}{9(1+\beta)}
\end{gather*}
As such, the Arrow-Debreu equilibrium is given by:
\begin{gather*}
    p_H=\beta^t \quad\mbox{and}\quad p_L=\beta^t\times\frac{9}{7} \quad\forall t\geq0 \\
    c_H^1=\frac{5+\frac{27}{7}\beta}{1+\beta}\\
    c_L^1=\frac{35+27\beta}{9(1+\beta)}\\
    c_H^2=\frac{\frac{36}{7}+4}{1+\beta}\\
    c_L^2=\frac{28+36\beta}{9(1+\beta)}
\end{gather*}

\newpage

\section{Exercise 3}
\subsection{a}

An Arrow-Debreu equilibrium is defined here as a household allocation decision (optimal consumption plan) $\{(c_t)\}^\infty_{t=0}$ and price system $\{(p_t)\}^\infty_{t=0}$ such that, given prices, the representative agents maximize utility subject to the budget constraint, resource constraint and the transversality condition, while the aggregate resource feasibility constraint is met (i.e. all of the markets clear).\\


Household's problem:
\begin{gather*}
    \max_{\{c_t^i\}^{+\infty}_{t=0}}\sum_{t=0}^{+\infty}\beta^tu(c_t^i)\quad\forall i\in I={1,2}
\end{gather*}

subject to the feasibility constraint:

\begin{gather*}
    \sum_{t=0}^{+\infty}p_tc_t^i\leq\sum_{t=0}^{+\infty}p_te_t^i
\end{gather*}

For household 1: $u(c_t^1)=c_t^1$ and for household 2: $u(c_t^2)=\ln(c_t^2$)\\

In addition,
\begin{gather*}
    e_t^1=\mu>0\quad\forall t\geq 0 \\
    e_t^2=0 \quad\mbox{when t is even}\\
    e_t^2=\alpha>0\quad\mbox{when t is odd}\\
    \mbox{where \hspace{0.1in}} \alpha=\mu\left(1+\frac{1}{\beta}\right)
\end{gather*}

and the market clearing condition:
\begin{gather*}
    \sum_{i\in I}c_t^i=\sum_{i\in I}e_t^i\quad\forall i\in I \\
    c_t^1+c_t^2=\mu\quad\mbox{when t is even}\\
        c_t^1+c_t^2=\mu+\alpha\quad\mbox{when t is odd}
\end{gather*}

\subsection{b}

First, we solve for the maximisation problem of consumer 1:

\begin{gather*}
    \max \sum^\infty_{t=0} \beta^tc^1_t
\end{gather*}

subject to

\begin{gather*}
  \sum^\infty_{t=0}  p_tc^1_t \leq \sum^\infty_{t=0} p_te_t^1 \mbox{\hspace{0.5in}} \forall t.
\end{gather*}

The Lagrangian of the problem can be written as

\begin{gather*}
    \mathcal{L} = \sum^\infty_{t=0} \beta^tc^1_t + \lambda^1 \left[ \sum^\infty_{t=0} p_te_t^1- \sum^\infty_{t=0} p_tc^1_t \right].
\end{gather*}

The first order conditions allow us to find the following results:

\begin{gather*}
\frac{\partial \mathcal{L}}{\partial c_t^1} = 0 \implies \beta^t = \lambda^1p_t \\
\frac{\partial \mathcal{L}}{\partial c_{t+1}^1} = 0 \implies \beta^{t+1} = \lambda^1p_{t+1} \\
\implies \frac{p_{t+1}}{p_{t}} = \beta.
\end{gather*}

Using the expression found above, and normalising $p_0 = 1$ we can solve for $p_t$:

\begin{gather*}
    p_0 = 1 \\
    p_1 = \beta \\
    p_2 = \beta^2 \\
    \vdots \\
    p_t = \beta^t.
\end{gather*}

Then we solve for the maximisation problem of consumer 2:

\begin{gather*}
    \max \sum^\infty_{t=0} \beta^t\ln(c^2_t)
\end{gather*}

subject to

\begin{gather*}
    p_tc^2_t \leq p_te_t^2 \mbox{\hspace{0.5in}} \forall t.
\end{gather*}

The Lagrangian of the problem can be written as

\begin{gather*}
    \mathcal{L} = \sum^\infty_{t=0} \beta^t\ln(c^2_t) + \lambda^2 \left[ \sum^\infty_{t=0} p_te_t^2 - \sum^\infty_{t=0} p_tc^2_t \right].
\end{gather*}

The first order conditions allow us to find the following results:

\begin{gather*}
    \frac{\partial \mathcal{L}}{\partial c_t^2} = 0 \implies \frac{\beta^t}{c^2_t} = \lambda^2p_t \\
    \frac{\partial \mathcal{L}}{\partial c_{t+1}^2} = 0 \implies \frac{\beta^{t+1}}{c^2_{t+1}} = \lambda^2p_{t+1} \\
    \implies \frac{p_{t+1}}{p_{t}} = \beta \frac{c^2_{t}}{c^2_{t+1}}.
\end{gather*}

Equating the two expressions found for $\frac{p_{t+1}}{p_t}$, we find that

\begin{gather*}
c^2_t = c^2_{t+1}.
\end{gather*}

From the budget constraint of consumer 2, we can recover $c^2_t$:

\begin{gather*}
    \sum \beta^tc^2_t = 0 + \beta^2\mu(1+\beta^{-1}) + 0 + \beta^3(1+\beta^{-1})+ \hdots \\
    \frac{c_t^2}{1-\beta} = \frac{\beta\mu(1+\beta^{-1})}{1-\beta^2}\\
    (1+\beta)c^2_t = \beta\mu(1+\beta^{-1})\\
    c^2_t = \frac{\beta\mu(1+\beta^{-1)}}{1+\beta}\\
    c^2_t = \mu
    \implies c^2_t = c^2_{t+1} = c^2 = \mu
\end{gather*}

From the resource constraint we can then find $c^1_t$ that for $t$ even:


Similarly, we can then find $c^1_t$ that for $t$ odd:


\subsection{c}

Time $t$ wealth is calculated as follows for consumer 1:

\begin{gather*}
    W_t^1 = \sum^\infty_{t = i} p_t(e^1_t-c^1_t)
\end{gather*}

Time $t$ wealth is calculated as follows for consumer 2:

\begin{gather*}
    W_t^2 = \sum^\infty_{t = i} p_t(e^2_t-c^2_t)
\end{gather*}

Suppose $i=0$, then wealths can be calculated as follows:

\begin{gather*}
    W_t^1 = \sum^\infty_{t = 0} p_t(e^1_t-c^1_t) \\
    = \beta^0\mu + \beta^1(\mu-\mu(1+\beta^{-1}) + \hdots
    = \frac{\mu}{1-\beta^2} - \frac{1}{1-\beta^2}
    = \frac{\mu - 1}{1-\beta^2}
\end{gather*}

\begin{gather*}
    W_t^2 = \sum^\infty_{t = i} p_t(e^2_t-c^2_t) \\
    = -\beta^0\mu + \beta^1(-\mu+\mu(1+\beta^{-1}) + \hdots
    = -\frac{\mu}{1-\beta^2} + \frac{1}{1-\beta^2}
    = \frac{1-\mu}{1-\beta^2}
\end{gather*}

We notice that $\forall t \geq 0$ it is true that $W_t^1 = -W^1_t$. This equality holds because the market clears in every period.

\subsection{d}

A sequential markets equilibrium is an allocation $\{(c_t, a_{t+1})\}^\infty_{t=0}$ and a price system $\{(q_t)\}^\infty_{t=0}$ such that the representative agents maximise utility subject to the budget constraint and resource feasibility constraint at each period $t$, the No-Ponzi scheme condition is not violated, and the aggregate resource feasibility constraint is met (i.e. markets clear).
Household's problem:
\begin{gather*}
    \max_{\{c_t^i\}^{+\infty}_{t=0}}\sum_{t=0}^{+\infty}\beta^tu(c_t^i)\quad\forall i\in I={1,2}
\end{gather*}

subject to the feasibility constraint:

\begin{gather*}
    \sum_{t=0}^{+\infty}p_tc_t^i\leq\sum_{t=0}^{+\infty}p_te_t^i
\end{gather*}

For household 1: $u(c_t^1)=c_t^1$ and for household 2: $u(c_t^2)=\ln(c_t^2$)\\

In addition,
\begin{gather*}
    e_t^1=\mu>0\quad\forall t\geq 0 \\
    e_t^2=0 \quad\mbox{when t is even}\\
    e_t^2=\alpha>0\quad\mbox{when t is odd}\\
    \mbox{where \hspace{0.1in}} \alpha=\mu\left(1+\frac{1}{\beta}\right)
\end{gather*}

and the market clearing condition:
\begin{gather*}
    \sum_{i\in I}c_t^i=\sum_{i\in I}e_t^i\quad\forall i\in I \\
    c_t^1+c_t^2=\mu\quad\mbox{when t is even}\\
        c_t^1+c_t^2=\mu+\alpha\quad\mbox{when t is odd}
\end{gather*}
The asset market clears:
\begin{gather*}
    \sum_{i\in I}c_t^i=\sum_{i\in I}a_t^i=0\quad\forall t\geq 0
\end{gather*}

Finally, the consumer do not overaccumulate capital
\begin{gather*}
    \lim_{t\rightarrow +\infty} p_ta_{t+1}^i \geq 0
\end{gather*}

We impose a non Ponzi scheme:
\begin{gather*}
    a_{t+1}^i \geq - \bar{A}
\end{gather*}
\subsection{e}

The maximisation problem of consumer 1 can be written as follows:

\begin{gather*}
    \max \sum^\infty_{t=0} \beta^tc^1_t 
\end{gather*}

subject to\footnote{We use a slightly different notation here compared to the one used in the other solutions (cf. question 2 and 4), treating $q_t$ as the price of the asset at time $t+1$. The solutions derived this way are equivalent had we used a formulation where we write $(1+r_t)a_t$.}

\begin{gather*}
    c^1_t + q_ta^1_{t+1} \leq e^1_t+a^1_t.
\end{gather*}

The Lagrangian of the problem can be written as

\begin{gather*}
    \mathcal{L} = \sum^\infty_{t=0} \left[ \beta^tc^1_t + \lambda_t^1[e^1_t + a^1_t - c^1_t -q_ta^1_{t+1}] \right].
\end{gather*}

The first order conditions allow us to find the following results:

\begin{gather*}
    \frac{\partial \mathcal{L}}{\partial c_t^1} = 0 \implies \beta^t = \lambda^1_t \\
    \frac{\partial \mathcal{L}}{\partial c_{t+1}^1} = 0 \implies \beta^{t+1} = \lambda^1_{t+1} \\
    \frac{\partial \mathcal{L}}{\partial a_{t+1}^1} = 0 \implies -\lambda_t^1q_t + \lambda^1_{t+1} = 0 \\
    \implies q_t = \frac{\lambda^1_{t+1}}{\lambda_t^1}  
    \implies q_t = \beta.
\end{gather*}

Meanwhile, the maximisation problem of consumer 2 is given by the following expression:

\begin{gather*}
    \max \sum^\infty_{t=0} \beta^t\ln(c^2_t) 
\end{gather*}

subject to

\begin{gather*}
    c^2_t + q_ta^2_{t+1} \leq e^2_t+a^2_t.
\end{gather*}

The Lagrangian of the problem can be written as

\begin{gather*}
    \mathcal{L} = \sum^\infty_{t=0} \left[ \beta^t\ln(c^2_t) + \lambda^2_t[e^2_t + a^2_t - c^2_t -q_ta^2_{t+1}] \right].
\end{gather*}

The first order conditions allow us to find the following results:

\begin{gather*}
    \frac{\partial \mathcal{L}}{\partial c_t^2} = 0 \implies \frac{\beta^t}{c^2_t} = \lambda^2_t \\
    \frac{\partial \mathcal{L}}{\partial c_{t+1}^2} = 0 \implies \frac{\beta^{t+1}}{c^2_{t+1}} = \lambda^2_{t+1} \\
    \frac{\partial \mathcal{L}}{\partial a_{t+1}^2} = 0 \implies -\lambda_t^2q_t + \lambda^2_{t+1} = 0 \\
    \implies q_t = \frac{\lambda^2_{t+1}}{\lambda_t^2}   \\
    \implies q_t = \beta \frac{c^2_t}{c^2_{t+1}}.
\end{gather*}

Equating the two expressions found for $q_t$ results in:

\begin{gather*}
    \beta = \beta\frac{c^2_t}{c^2_{t+1}} \\
    \implies c_t^2 = c_{t+1}^2 = c^2.
\end{gather*}

Then we can find expressions for $c^2$ recursively as follows:

\begin{gather*}
    t = 0 \mbox{\hspace{0.3in}} c_0^2 + q_0a_1^2 = 0 \\
    t = 1 \mbox{\hspace{0.3in}} c_1^2+q_1a_2^2 = \mu(1+\beta^{-1})+a^2_1 \\
    \implies a^2_1 = c^2_1+q_1a^2_2-\mu(1+\beta^{-1}) \\
    \implies c^2_0 + q_0(c_1^2 + q_1a^2_2-\mu(1+\beta^{-1})) = 0 \\
    \implies c^2_0 + q_0c_1^2 + q_oq_1a^2_2 = q_0\mu(1+\beta^{-1})) \\
    t = 2 \mbox{\hspace{0.3in}} c_2^2 + q_2a_3^2 = 0 + a^2_2 \\
    \implies c_0^2 + q_0c_1^2 + q_0q_1c_2^2 + q_0q_1q_2a^2_3 = q_0\mu(1+\beta^{-1}) \\
    \vdots
\end{gather*}

and so on and so forth. Ultimately, we would find the following relationship:

\begin{gather*}
    \sum_{t=0}^\infty \beta^tc^2_t + \prod^\infty_{t=0} \beta^t a^2_t =  \beta\mu(1+\beta^{-1}) + \beta^3\mu(1+\beta^{-1}) + \hdots
\end{gather*}

where as $t \rightarrow \infty$, $\prod^\infty_{t=0} \beta^t a^2_t \rightarrow 0$. Then:

\begin{gather*}
    \sum_{t=0}^\infty \beta^tc^2_t =  \beta\mu(1+\beta^{-1}) + \beta^3\mu(1+\beta^{-1}) + \hdots \\
    \frac{c^2_t}{1-\beta} = \frac{\beta\mu(1+\beta^{-1})}{1-\beta^2} \\
    (1+\beta)c^2_t = \mu(1+\beta) \\
    c^2_t = \mu \mbox{\hspace{0.1in}} \forall t \geq 0.
\end{gather*}

From the resource constraint, we recover the $c^1_t$ for the even and odd periods. \\

For even periods:

\begin{gather*}
    t = 0, 2, 4, \hdots \\
    c^1_t + c^2_t = \mu \\
    c^1_t = 0.
\end{gather*}

For odd periods:

\begin{gather*}
    t = 1, 3, 5, \hdots \\
    c^1_t + c^2_t = \mu + \mu(1+\beta^{-1}) \\
    c^1_t = \mu(1+\beta^{-1}).
\end{gather*}

This is what we found in the previous question, establishing that the sequential market equilibrium is equivalent to the Arrow-Debreu equilibrium. \\

In addition, it can be inferred that in even periods, consumer 1 gives to consumer 2 and vice versa in odd periods. Assume that $a^1_0 = a^2_0 = 0$, then for consumer 1:

\begin{gather*}
    0 + a^1_1 = \mu 
    \implies a^1_1 = \mu \\
    \mu(1+\beta^{-1}) + \beta a_2^1 = \mu + a^1_1 \\
    \mu(1+\beta^{-1}) + \beta a_2^1 = \mu + \mu \\
    a_2^1 = \frac{\mu}{\beta}-\frac{\mu}{\beta^2} \\
    \vdots
\end{gather*}

so on and so forth.\\

And for consumer 2:

\begin{gather*}
    \mu + a^2_1 = 0 
    \implies a^2_1 = -\mu \\
    \mu + \beta a_2^2 = \mu(1+\beta^{-1}) + a^2_1 \\
    \mu + \beta a_2^2 = \mu + \frac{\mu}{\beta} -\mu \\
    a_2^1 = \frac{\mu}{\beta^2}-\frac{\mu}{\beta} \\
    \vdots
\end{gather*}

so on and so forth.\\

Hence we find that:

\begin{gather*}
    a^1_1 = \mu \mbox{\hspace{0.3in}} a^2_1 = -\mu \\
    a^1_2 = \frac{\mu}{\beta}-\frac{\mu}{\beta^2} \mbox{\hspace{0.3in}} a^2_2 = \frac{\mu}{\beta^2}-\frac{\mu}{\beta} \\
    \vdots
\end{gather*}

so on and so forth. Indeed, since asset markets clear, $a^1_t = -a^2_t$.

\color{black}

\newpage

\section{Exercise 4}
\subsection{a}
The entrepreneur wants to maximise their profit by choosing the optimal level of capital each period. Households provide labor inelastically.

\begin{gather*}
    \max_{k_t} y_t-k_tR_t\quad\forall t\geq 0
\end{gather*}

With $y_t=f(k_t)=k_t^{\alpha}$, the output produced using the production function $f$ and $R_t=1+r_t$, the return of capital ($r_t$ being the interest rate). \\

FOC:

\begin{gather*}
    \frac{\partial \mathcal{L}}{\partial k_t}=0 \Leftrightarrow R_t=\alpha k_t^{\alpha-1}
\end{gather*}

As such, $R_t=1+r_t=\alpha k_t^{\alpha-1}$: the the return of capital should be equal to the marginal productivity of capital. We can deduce that: $r_t=\alpha k_t^{\alpha-1}-1$

The entrepreneur has borrowed from the household to buy the capital used next period. Indeed, from the law of motion of capital:

\begin{gather*}
    k_{t+1}=(1-\delta)k_t+i_t\quad\mbox{with}\quad\delta=1\quad\mbox{because capital fully depreciates}\\
    k_{t+1}=i_t
\end{gather*}

As such, the entrepreneurs consume the profit made (amount produced $f(k_t)=y_t$) minus what he has to pay to the household ($R_tk_t$). 

\begin{gather*}
    c_t=y_t-r_tk_t-k_t\\
    c_t=k_t^{\alpha}-(\alpha k_t^{\alpha-1}-1)k_t -k_t \\
    c_t=(1-\alpha)k_t^{\alpha}
\end{gather*}

\subsection{b}

The goods market equilibrium is reached when the supply for a good is equal to the demand for that good. \\

Households have an endowment which can be used to either consume or save.\\
Because the total endowment is always 1, every period the aggregate budget constraint is:

\begin{gather*}
    c_t^A+c_t^B+a_{t+1}^A+a_{t+1}^B=1+(1+r_t)(a_t^A+a_t^B)\quad(2)\quad\forall t\geq 0
\end{gather*}

%\textcolor{red}{I'm not convinced that Eq. 2 is correctly specified. The problem states that the assets that are saved in period t are also denoted with subscript t. Given that the left-hand side of the equation is meant to describe the collective expenses of the households (which are divided between consumption and saving/investment), the assets should be subscripted with t. Furthermore, the right-hand side describes the collective resources that the two households expend; it seems correct to include 1, since this is the aggregate endowment in the period, but I'm not convinced that the second element that's multiplied by the interest rate is correct. If the intention is to say that the household also expends the payoff from the savings that they invested in the previous period, then those assets should be subscripted in period t-1, and they should be receiving both the principal and the interest in order for this answer to be consistent with what has been written for the previous question. And, in any case, this question asks about the market for the consumer good, and not for the asset that is saved/invested. It seems like Eq. 2 has veered into the realm of describing a budget constraint, rather than a market clearing condition, which is what the question is looking for I think. I think that is the reason why the market clearing condition that is given in the problem does not explicitly state the saving/investment that is conducted.  }
The amount saved for the next period ($a_{t+1}^A+a_{t+1}^B$) will be used by the entrepreneur to buy capital and produce the final good. Each period, the entrepreneur has to repay what was borrowed previously plus the interest rates on this amount $(a_{t}^A+a_{t}^B)(1+r_t)$. Total flows from creditors (households) and borrowers (entrepreneurs) should be equal at the equilibrium. Otherwise, it would mean that the entrepreneurs could borrow until infinity. At the limit, both flows compensate each other and the good produced by the entrepreneur is either used to consume or buy capital (assuming capital fully depreciates every period). Because the sum of flows is 0 at the limit, assets do not appear in the equation and subscripts for $c$, $k$ and $f(k)$ are removed. Using the aggregate budget constraint of the households described before, adding k and c on the left side and $f(k)$ on the right side include both flows. \\

Finally, the sum of the good produced $f(k)$ and the endowment each period are equal to the lifetime consumption and capital bought by the entrepreneurs plus the consumption at each period from the two agents A and B.\\

Therefore:
\begin{gather*}
    c_t^A+c_t^B+c+k=f(k)+1
\end{gather*}


\color{black}
An additional interpretation would be as follows, in the case of there being no credit constraint. The results we use in this explanation will be formally proved in the next section. Each consumer receives a different endowment at each time period as compared to their peer. However, as we've seen, the agents have identical utility functions and they face essentially the same budget constraint in the limit, so the consumption-saving decision that Agent A makes in period t is essentially identical to the consumption-saving decision that Agent B will make in period t+1. The consistency of this conclusion is further reinforced by the fact that we know that both agents will smooth consumption over time. But, from the perspective of the firm, which interacts with both agents simultaneously in each period, they will be receiving the same supply of the asset in each period, independent of which agent happens to have received an endowment at that point. So, it would seem to follow that the firm's optimal decision in any given period t between payment for lent capital and consumption of surplus output would be identical to their optimal decision in any further period t+1. Given that the firm's technology does not change and the agents will collectively present the firm with an identical supply of capital each period, one can confidently remove the subscripts from the consumption of the firm (c) and the production of the firm ($f(k)$), as they will be identical each period. The invested consumption good in the form of capital also remains constant in each period and is denoted as such for the same reasons.



\subsection{c}

\subsubsection{i}
We write the program for the rich agent, assuming $i$ is rich at $t$ (respectively in the even period for A and the odd period for B).\\
%\textcolor{red}{I think we should be a bit more specific with our notation. The agents here are denoted $i \in {1,2}$ but the problem consistently refers to them as Agents A and B, or Rich and Poor. I think it would be really important to change the indexing of the problem below to one of these two.} \\

In a sequential market equilibrium, agents choose every period their optimal level of consumption and assets $\{(c_t, a_{t+1}\}^{+\infty}_{t=0}$. The one period interest rate is $r_t$. The households maximise their utility subject to their budget constraint at each period $t$ and the feasibility condition is met (i.e. all of the markets clear).\\

$\forall i=\{A,B\}$, the maximization problem is the following:

\begin{gather*}
    \max_{\{c_t^i, a_{t+1}^i\}^{+\infty}_{t=0}}\sum_{t=0}^{\infty}u(c_t^i)\quad\forall t \geq 0
\end{gather*}

subject to the budget constraint:

\begin{gather*}
    c_t^i+a^i_{t+1}\leq e_t^i+(1+r_t)a_t^i.
\end{gather*}

We maintain that the agent is rich in period $t$ (and poor in period $t+1$). The agent acquires assets $a_t$ in the previous period (when poor):

\begin{gather*}
    c_t^R+a^R_{t+1}\leq 1+(1+r_t)a_t^P
\end{gather*}
In the next period $t+1$, the agent is poor:\\
\begin{gather*}
    c_{t+1}^P+a^P_{t+2}\leq 1+(1+r_{t+1})a_{t+1}^R
\end{gather*}
The budget constraint should be binding because the utility function is strictly increasing in consumption, and the household has an interest in consuming and saving for future consumption all of his endowment in every period.\\

%We impose a no-Ponzi scheme, which is a lower bound to the amount the agent can borrow from the other agent. Otherwise, the amount borrowed could go to $-\infty$ and consumption would be infinite. Each agent should be able to repay their debt.

In addition, the credit constraint is not binding (households are not credit constrained):

\begin{gather*}
    a_{t+1}^i\geq0\quad\forall i\in\{P,R\} \mbox{\hspace{0.3in}} \forall t \geq 0 \\
\end{gather*}

The initial condition of assets at the beginning of time is taken to be as follows:
\begin{gather*}
    a^i_{0} = 0
\end{gather*}

The feasibility constraint is met. From b), the goods market should clear: $c_t^A+c_t^B+k+c=1+f(k)$. \\
Finally, the transversality condition holds
\begin{gather*}
    \lim_{t\rightarrow +\infty} p_ta_{t+1}^i \geq 0
\end{gather*}

%\textcolor{red}{I'm not convinced that the Non-Ponzi condition is correctly specified. The problem states that the following condition never binds: $a^R_{t+1} \geq 0$. This essentially says that, when the agent is rich, he will always engage in a strictly positive level of savings. That is, when the agent is rich, he will never *not save* and he will never borrow. The answer that is written above says something different. What it essentially says is that *both* agents (ie the Rich and Poor agents) simultaneously, in period t, will engage in non-negative savings. This is different from the condition given to us by the problem because it refers to both the Rich and Poor agents. Also, what is written above would seem to violate the market clearing condition for assets; it may be possible for a Rich agent at period t to save a part of their endowment and sell it to the firm, but would a Poor agent with no endowment at period t be able to save and sell to the firm? Wouldn't the Poor agent instead be acquiring assets from the Rich agent, as is said in the beginning of the answer to this problem? \\
%I think it may be worth restating the condition provided by the problem that refers to the savings of the Rich agent at period t, and then supplementing this with a statement of the Non-Ponzi condition which is applicable to any agent i in period t. \\
%Maybe it's also worth noting that there must be an initial condition for assets in order for the equilibrium stream to exist: i.e. $a^i_{0} = 0, \forall i \in {R, P}$}


\subsubsection{ii}
%\textcolor{red}{Assets in the first part of the Lagrangian should be superscripted with a P to denote the Rich state of the agent.} \\
We define the Lagrangian as follows, taking agent A rich in the even period:


The constraint $a_{t+1}^i\geq 0$ is not binding for rich and poor agents (they are not credit constrained). As such, by complementary slackness, $\mu^i_{t}=0$.
We define the indicator $\mathbf{1}\{t\equiv0[2]\}$ which is equal to 1 when $t$ is even and 0 otherwise (i.e. when A is rich) and where $a[b]$ is the modulus. Similarly, $\mathbf{1}\{t\equiv1[2]\}$ is equal to 0 when $t$ is even and 1 otherwise (i.e. when A is poor).\\

FOCs: \\

If $t\equiv0[2]$, and $t+1\equiv1[2]$\\

\begin{gather*}
    \frac{\partial \mathcal{L}}{\partial c_t^R}=0\Leftrightarrow u'(c_t^R)=\lambda_t^R \\
    \frac{\partial \mathcal{L}}{\partial a_{t+1}^R}=0\Leftrightarrow \beta^t\lambda_{t}^R=\beta^{t+1}\lambda_{t+1}^P(1+r_{t+1})\Leftrightarrow \lambda_{t}^R=\beta\lambda_{t+1}^P(1+r_{t+1})
\end{gather*}

We combine the two FOCs and get:

\begin{gather*}
    u'(c_{t}^R)=\beta\lambda_{t+1}^P(1+r_{t+1})
\end{gather*}

The same reasoning can be applied to agent B by changing the indicator in the Lagrangian. Indeed, we define the indicator $\mathbf{1}\{t\equiv0[2]\}$ which is equal to 0 when $t$ is even and 1 otherwise (i.e. when B is rich). Similarly, $\mathbf{1}\{t\equiv1[2]\}$ is equal to 1 when $t$ is even and 0 otherwise (i.e. when B is poor). This leads to the same results:

\begin{gather*}
    u'(c_t^R)=\lambda_t^R \\
    u'(c_{t}^R)=\beta\lambda_{t+1}^P(1+r_{t+1})
\end{gather*}

Which is the Euler equation when the agent is rich. \\
%\textcolor{red}{In principle, the Lagrangian schemes that are presented aren't wrong and they yield the correct answer; but, technically speaking, this answer proposes two different programs: one for Agent A and one for Agent B. However, the problem asked us to present a *single* program for the Rich agent. In fact, the demand for such a program is substantiated in the homework by the fact that the programs of Agents A and B are exactly identical when examined in the relevant time periods. I think this could also be the reason why the question asks us to define the wealth at the beginning of the period as $q_t$; they may be providing us with the variable to be used in the indicator function. \\
%So, the change that I propose is to make the indicator function contingent on the value of the endowment, rather than the time period. As we've seen, if the indicator function is based on the time period, then one is forced to write two different programs for each agent. However, if you make the indicator function contingent on the value of the endowment at each time period, then the indicator function doesn't have to be changed when one is switching from analyzing Agent A to analyzing Agent B, and vice versa. It would just so happen that the indicator function takes a different sequence of values depending on the time period by virtue of the agent's endowment stream that we know a priori.}

%\textcolor{red}{We take the consumer to be rich in period t and poor in period t+1. The Lagrangian associated to the problem of the rich consumer above is as follows: \\
%\begin{gather*}
%\mathcal{L} = \sum_{t=0}^{\infty} \beta^t [(u(c^R_t) - \lambda^R_t(c^R_t + a^R_{t+1} - 1 - (1+r_t)a^P_t)\textbf{1}_{1}(e^i_t) + \\
%(u(c^P_t) - \lambda^P_t(c^P_t + a^P_{t+1} - (1+r_t)a^R_t)\textbf{1}_0(e^i_t)], \forall i \in {A,B}
%\end{gather*}
%The indicator function $\textbf{1}_{1}(e^i_t)$ takes the value 1 when the endowment of agent $i$ is 1, ie when agent $i$ is rich, and it takes the value 0 when the endowment of agent $i$ is 0, ie when the agent is poor. Similarly, the indicator function $\textbf{1}_0(e^i_t)$ takes the value 1 when the endowment of the agent $i$ is 0, ie when the agent $i$ is poor, and it takes the value 0 when the endowment of agent $i$ is 1, ie when the agent is rich. There is no need to specify a specific agent $i$ when the First Order Conditions are calculated, because the Lagrangian is written for the general case of the rich agent. It is enough to posit that the agent is taken to be rich at period t and poor at period t+1. Therefore, the First Order Conditions remain the same as they are written above; it follows that the Euler equation is identical as well.}






\subsection{d}

\subsubsection{i}

We write the program for the poor agent, assuming $i$ is poor at $t$ (in the odd period for A and the even period for B, respectively). \\

In a sequential market equilibrium, agents choose in every period their optimal level of consumption and assets $\{(c_t, a_{t+1}\}^{+\infty}_{t=0}$. The one period interest rate is $r_t$. The households maximise their utility subject to their budget constraint in each period $t$ and the feasibility condition is met (i.e. all of the markets clear).\\

$\forall i=\{A,B\}$, the maximization problem is the following:

\begin{gather*}
    \max_{\{c_t^i, a_{t+1}^i\}^{+\infty}_{t=0}}\sum_{t=0}^{\infty}u(c_t^i)\quad\forall t \geq 0
\end{gather*}

subject to the budget constraint:

\begin{gather*}
    c_t^i+a^i_{t+1}\leq e_t^i+(1+r_t)a_t^i
\end{gather*}

We maintain that the agent is poor in period $t$ (and rich in period $t+1$). The agent buys $a_t$ in the previous period (when he is rich):

\begin{gather*}
    c_t^P+a^P_{t+1}\leq (1+r_t)a_t^R
\end{gather*}
In the next period $t+1$, the agent is rich:\\
\begin{gather*}
    c_{t+1}^R+a^R_{t+2}\leq 1+(1+r_{t+1})a_{t+1}^P
\end{gather*}
The budget constraint should be binding because the utility function is strictly increasing in consumption, and the household has an interest in consuming and saving for future consumption all of his endowment in every period.\\

In addition, the credit constraint is not binding:

\begin{gather*}
    a_{t+1}^i\geq0\quad\forall i\in\{P,R\} \mbox{\hspace{0.3in}} t \geq 0 \\
\end{gather*}
The initial condition of assets at the beginning of time is taken to be as follows:

\begin{gather*}
    a^i_{0} = 0
\end{gather*}

The feasibility constraint is met. From b), the good market should clear: \\ 
$c_t^A+c_t^B+k+c=1+f(k)$. \\
Finally, the transversality condition holds:
\begin{gather*}
    \lim_{t\rightarrow +\infty} p_ta_{t+1}^i \geq 0
\end{gather*}

%\textcolor{red}{All of the same comments that I made in Part c(i) hold for Part d(i), assuming that they are relevant in the former case.}

\subsubsection{ii}
We define the Lagrangian as follows, taking agent A poor in odd period:
\begin{gather*}
    L=\sum_{t=0}^{+\infty}\beta^t ( \left( u(c_t^R)-\lambda_t^R(c_t^R+a^R_{t+1}-1-(1+r_t)a_t^P)+a_{t+1}^R\mu^R_{t})\mathbf{1}\{t\equiv0[2]\}\right)\\
    +\left( u(c_t^P)-\lambda_t^P(c_t^P+a_{t+1}^P-(1+r_t)a_t^R)+a_{t+1}^P\mu^P_{t})\mathbf{1}\{t\equiv1[2]\}\right) )
\end{gather*}
The constraint $a_{t+1}^i \geq 0$ is not binding for rich and poor agents (they are not credit-constrained). As such, by complementary slackness, $\mu^i_{t}=0$.\\
We define the indicator $\mathbf{1}\{t\equiv0[2]\}$ which is equal to 1 when $t$ is even and 0 otherwise (i.e. when A is rich). Similarly, $\mathbf{1}\{t\equiv1[2]\}$ is equal to 0 when $t$ is even and 1 otherwise (i.e. when A is poor).\\

FOCs: If $t\equiv1[2]$ (A is poor at $t$), and $t+1\equiv0[2]$ (A is rich at $t+1$)

\begin{gather*}
    \frac{\partial \mathcal{L}}{\partial c_t^P}=0\Leftrightarrow u'(c_t^P)=\lambda_t^P \\
    \frac{\partial \mathcal{L}}{\partial a_{t+1}^P}=0\Leftrightarrow \beta^t\lambda_{t}^P=\beta^{t+1}\lambda_{t+1}^R(1+r_{t+1})\Leftrightarrow \lambda_{t}^P=\beta\lambda_{t+1}^R(1+r_{t+1})
\end{gather*}

We combine the two FOCs and get:

\begin{gather*}
    u'(c_{t}^P)=\beta\lambda_{t+1}^R(1+r_{t+1})
\end{gather*}

Which is the Euler equation when the agent is poor.\\

The same reasoning can be applied to agent B by changing the indicator in the Lagrangian. Indeed, we define the indicator $\mathbf{1}\{t\equiv0[2]\}$ which is equal to 0 when $t$ is even and 1 otherwise (i.e. when B is rich). Similarly, $\mathbf{1}\{t\equiv1[2]\}$ is equal to 1 when $t$ is even and 0 otherwise (i.e. when B is poor). This leads to the same results:
\begin{gather*}
    u'(c_t^P)=\lambda_t^P \\
    u'(c_{t}^P)=\beta\lambda_{t+1}^R(1+r_{t+1})
\end{gather*}

%\textcolor{red}{All of the same comments that I made in part c(ii) hold for Part d(ii), assuming that they are relevant in the former case.}

%\textcolor{red}{We take the consumer to be poor in period t and rich in period t+1. The Lagrangian associated to the problem of the rich consumer above is as follows: \\
%\begin{gather*}
%\mathcal{L} = \sum_{t=0}^{\infty} \beta^t [(u(c^R_t) - \lambda^R_t(c^R_t + a^R_{t+1} - 1 - (1+r_t)a^P_t)\textbf{1}_{1}(e^i_t) + \\
%(u(c^P_t) - \lambda^P_t(c^P_t + a^P_{t+1} - (1+r_t)a^R_t)\textbf{1}_0(e^i_t)], \forall i \in {A,B}
%\end{gather*}
%The indicator function $\textbf{1}_{1}(e^i_t)$ takes the value 1 when the endowment of agent $i$ is 1, ie when agent $i$ is rich, and it takes the value 0 when the endowment of agent $i$ is 0, ie when the agent is poor. Similarly, the indicator function $\textbf{1}_0(e^i_t)$ takes the value 1 when the endowment of the agent $i$ is 0, ie when the agent $i$ is poor, and it takes the value 0 when the endowment of agent $i$ is 1, ie when the agent is rich. There is no need to specify a specific agent $i$ when the First Order Conditions are calculated, because the Lagrangian is written for the general case of the poor agent. It is enough to posit that the agent is taken to be poor at period t and rich at period t+1. Therefore, the First Order Conditions remain the same as they are written above; it follows that the Euler equation is identical as well. 
%}







\subsection{e}
At the steady state: \\
%\textcolor{red}{Maybe we can superscript the first two constants with their respective states, as opposed to subscripting them?}
\begin{gather*}
    c_t^P=c^P\\
    c_t^R=c^R\\
    \lambda^R_t=\lambda^R_{t+1}=\lambda^R\\
    \lambda^P_t=\lambda^P_{t+1}=\lambda^P\\
    r_t=r_{t+1}=r
\end{gather*}

We use the 4 equations we found in c) and d), which become at the steady state:
\begin{gather*}
    u'(c^P)=\lambda^P\quad (1)\\
    u'(c^P)=\beta\lambda^R (1+r)\quad (2)\\
    u'(c^R)=\lambda^R\quad (3)\\
    u'(c^R)=\beta\lambda^P (1+r)\quad (4)
\end{gather*}

In addition: $\lambda^R=\lambda^P=\lambda$ because, dividing (2) and (4) and replacing $u'(c^P)=\lambda^P$ and $u'(c^R)=\lambda^R$:

\begin{gather*}
    \frac{\lambda^R}{\lambda^P} = \frac{\lambda^P}{\lambda^R} \\
    (\lambda^{P})^2 = (\lambda^{R})^2 \\
    \implies \lambda^P = \lambda^R
\end{gather*}
Because $\lambda^P, \lambda^R>0$ \\

Then, combining (1) and (3), we get
\begin{gather*}
    \lambda^R=\lambda^R(1+r)\beta\Leftrightarrow 1+r=\frac{1}{\beta}
\end{gather*}

Combining this result and (4):
\begin{gather*}
    u'(c^R)=\lambda^P\times\beta\times\frac{1}{\beta}=\lambda^P
\end{gather*}
Combining this result and (1):
\begin{gather*}
    u'(c^R)=u'(c^P)\Leftrightarrow c^R=c^P=c
\end{gather*}
Assuming the utility function is strictly concave.

\subsection{f}

The RBC is a neoclassical growth model with stochastic productivity growth and endogenous labour supply. The canonical RBC model has 3 goods (output, labour, capital) and has perfect competition as well as perfect information. While the RBC model has many different extensions, we will be comparing the current model with this canonical, simple RBC model for clarity. \\

In the RBC model, the saving rate is constant over time, and the main source of output fluctuation is the technological shock. On the contrary, in this model,  fluctuations in output depend on changes in the saving rate which depend on the fluctuating level of endowments, which directly depend on whether the agent is "rich" or "poor". Thus the primary difference between the RBC model and the current model is that there is heterogeneity in endowments at any one period $t$ contrary to the RBC model in which consumption of agents is driven by production. The focus of the canonical RBC model is to understand how exogenous productivity shocks affect the economy while it is at the steady state. Meanwhile, when studying models with heterogeneous agents, our focus is more on understanding the evolution of consumption and wealth of agents. Indeed, in this model, agents can use savings to smooth consumption over time and in this case, there are no fluctuations in consumption.\\

It's important to highlight that the RBC model is the canonical model within the general "Representative Agent" paradigm. The primary impetus behind the creation of the RBC model was to model the business cycle with as much mathematical parsimony as possible. So, the RBC model takes only a single agent as its category of analysis, and this one representative agent also happens to own and operate the single representative firm that produces within the economy. Information is everywhere and always perfect. The primary impetus behind introducing a heterogenous agent paradigm was to study certain stylized facts which were entirely absent within the representative agent paradigm, namely some heterogeneity in wealth. So, rather than containing a single representative agent, heterogenous agent models can have an infinite spectrum of agents, all of which have unique specifications of a certain characteristic (i.e. endowments, preferences, etc.). This categorical shift introduces very relevant economic frictions which come to the fore when analyzing the real economy empirically. \\

\subsection{g}

Although endowments are reduced to $e_t = 1-T$, the calculations presented above to derive the long-term interest rate remain the same. Therefore, at the steady-state, the following relationship still holds: 

\begin{gather*}
    r = \frac{1-\beta}{\beta}
\end{gather*}

Therefore, level-changes to the endowment level should not change the long-term interest rate since the endowment does not enter the above expression. Moreover, since we do not tax savings or consumption specifically, the opportunity cost of savings remains the same in the long-run. Indeed, the Euler equation remains the same, therefore the marginal benefit of consumption still equals the marginal cost of saving. 

%\textcolor{red}{Yes, the interest rate is entirely dependent on the discount factor, which is a factor that we take to be constant at all times. Given that the tax only acts to detract from the Rich agent's endowment, it does not affect the structure of the agent's preferences for future consumption and, thus, does not affect the interest rate. \\
%The idea behind Ricardian equivalence, to put it briefly, is that two different regimes of taxation can produce two sequential market equilibriums that are identical, ie they are defined by the same sequence of consumption and savings (assets), as long as these tax regimes satisfy a certain relationship. This relationship consists of a zero-sum perspective on tax regimes. That is, if the government lowers taxes in the current period, then it must be the case that the government will increase taxes in the next period in order to pay off the bonds that were issued in order to finance government spending in the wake of the primary tax cut. The relationship is analogous if the government first increases taxes. The idea is that the consumer is forward-looking and has complete knowledge of the fact that the government conducts its spending and taxation, and balances its budget, in this manner. So, although an initial tax cut would seemingly incentivize an increase in consumption in the same period given that the consumer comes upon a greater-than-expected endowment, the consumer actually maintains their former level of consumption because they anticipate that the present surplus will have to be utilized to pay higher taxes in the future period. \\
%Looking at the present problem in particular: the relationship of the interest rate to the constant discount factor is established by two elements of the model: (1) the market clearing condition for goods and (2) the Euler Equation, which relates the ratio of consumption in the present and future periods to the product of the discount factor and return to capital. One could argue that a tax that is implemented on the endowments of both agents when they are Rich essentially acts as a linear transformation of the aggregate endowments (taking both agents together). So, although the agents find themselves in a context in which they have fewer resources at their disposal, they are still going to engage in consumption smoothing, as this is determined by the nature of their preferences. They will continue to borrow against each others' endowments, and will maintain the ratio of present to future consumption that persisted before the tax was implemented. \\
%However, one could similarly argue for the contrary. The tax regime that is proposed does not appear to be one that maintains the necessarily relationship for Ricardian Equivalence, ie the government is said to take resources from each agent at a certain period but is not said to balance this tax policy in the future.}

\subsection{h}
Now we assume that agent A in the odd period and agent B in the even period are credit constrained. \\

We write the program for the poor agent, assuming $i$ is poor at $t$ (respectively in the odd period for A and the even period for B): \\

In a sequential market equilibrium, agents choose every period their optimal level of consumption and assets $\{(c_t, a_{t+1}\}^{+\infty}_{t=0}$. The one period interest rate is $r_t$. However, in this case, when the agent is poor, he cannot choose the optimal level of the asset and must choose 0 by default. The households maximise their utility subject to their budget constraint at each period $t$ and the feasibility condition is met (i.e. all of the markets clear).\\

$\forall i=\{A,B\}$, the maximization problem is the following:\\
%\textcolor{red}{Again, the agents really shouldn't be denoted as numbers, but rather as A or B, or Rich or Poor.}
\begin{gather*}
    \max_{\{c_t^i, a_{t+1}^i\}^{+\infty}_{t=0}}\sum_{t=0}^{\infty}u(c_t^i)\quad\forall t \geq 0
\end{gather*}

subject to the budget constraint:

\begin{gather*}
    c_t^i+a^i_{t+1}\leq e_t^i+(1+r_t)a_t^i
\end{gather*}

When the agent is poor in period $t$ (and rich in period $t+1$). The agent acquires asset $a_t$ in the previous period (when he is rich):

\begin{gather*}
    c_t^P+a_{t+1}^P\leq (1+r_t)a_t^R
\end{gather*}

In the next period, the agent is rich at $t+1$:

\begin{gather*}
    c_{t+1}^R+a_{t+2}^R\leq 1 +(1+r_{t+1})a_{t+1}^P
\end{gather*}

The budget constraints should be binding because the utility function is strictky increasing in consumption, and the household has an interest in consuming or saving for future consumption all of his endowment in every period.\\

The credit constraint is binding when the agent is poor (credit constrained):

\begin{gather*}
a_{t+1}^P=0\quad\forall i\in\{A,B\}
\end{gather*}

The credit constraint is not binding when the agent is rich (not credit constrained).

\begin{gather*}
    a_{t+1}^R\geq0\quad\forall i\in\{A,B\}
\end{gather*}

The feasibility constraint is met. From b), the goods market should clear: $c_t^A+c_t^B+k+c=1+f(k)$. \\
Finally, the transversality condition holds
\begin{gather*}
    \lim_{t\rightarrow +\infty} p_ta_{t+1}^i \geq 0
\end{gather*}

We define the Lagrangian as follows, taking agent A poor in odd period:
%\textcolor{red}{The asset variable in the budget constraint of the Rich lacks a superscript denoting it as such. Also, we've removed $a^P_{t+1}$ but we have not removed $a^P_{t}$? If the agent is credit constrained when they are poor, shouldn't that be the case for all periods in which he is poor?}
\begin{gather*}
    L=\sum_{t=0}^{+\infty}\beta^t ( \left( u(c_t^R)-\lambda_t^R(c_t^R+a^R_{t+1}-1-(1+r_t)a^P_t)+\mu^R_t a_{t+1}^R)\mathbf{1}\{t\equiv0[2]\}\right)\\
    +\left( u(c_t^P)-\lambda_t^P(c_t^P+a^P_{t+1}-(1+r_t)a_t^R)+\mu^P_t a_{t+1}^P))\mathbf{1}\{t\equiv1[2]\}\right) )
\end{gather*}
Because the credit constraint is not binding for the rich agent, by complementary slackness: $\mu_t^R=0$ for all $t$. On the contrary, because the credit constraint is binding for the poor agent, by complementary slackness: $\mu_t^P>0$.
Indeed, we define the indicator $\mathbf{1}\{t\equiv0[2]\}$ which is equal to 1 when $t$ is even and 0 otherwise (i.e. when A is rich). Similarly, $\mathbf{1}\{t\equiv1[2]\}$ is equal to 0 when $t$ is even and 1 otherwise (i.e. when A is poor).\\

%\textcolor{red}{We take the consumer to be poor in period t and rich in period t+1. Also, the consumer is assumed to be credit constrained when they are poor. The Lagrangian associated to the problem of the rich consumer above is as follows: \\
%\begin{gather*}
%\mathcal{L} = \sum_{t=0}^{\infty} \beta^t [(u(c^R_t) - \lambda^R_t(c^R_t + a^R_{t+1} - 1)\textbf{1}_{1}(e^i_t) + \\
%(u(c^P_t) - \lambda^P_t(c^P_t - (1+r_t)a^R_t)\textbf{1}_0(e^i_t)], \forall i \in {A,B}
%\end{gather*}
%The indicator function $\textbf{1}_{1}(e^i_t)$ takes the value 1 when the endowment of agent $i$ is 1, ie when agent $i$ is rich, and it takes the value 0 when the endowment of agent $i$ is 0, ie when the agent is poor. Similarly, the indicator function $\textbf{1}_0(e^i_t)$ takes the value 1 when the endowment of the agent $i$ is 0, ie when the agent $i$ is poor, and it takes the value 0 when the endowment of agent $i$ is 1, ie when the agent is rich. There is no need to specify a specific agent $i$ when the First Order Conditions are calculated, because the Lagrangian is written for the general case of the poor agent. It is enough to posit that the agent is taken to be poor at period t and rich at period t+1. Therefore, the First Order Conditions remain the same as they are written above; it follows that the Euler equation is identical as well. 
%}




FOCs: If $t\equiv1[2]$ (A is poor at $t$), and $t+1\equiv0[2]$ (A is rich at $t+1$)

\begin{gather*}
    \frac{\partial \mathcal{L}}{\partial c_t^P}=0\Leftrightarrow u'(c_t^P)=\lambda_t^P\\
     \frac{\partial \mathcal{L}}{\partial a_{t+1}^P}=0\Leftrightarrow \beta^t\lambda_t^P=\beta^{t+1}(\lambda_{t+1}^R(1+r_{t+1})+\mu_{t}^P)\Leftrightarrow \lambda_t^P=\beta\lambda_{t+1}^R(1+r_{t+1})+\mu_{t}^P
\end{gather*}

Which is the Euler equation when the agent is poor.\\

On the contrary, if $t\equiv1[2]$ (A is rich at $t$), and $t+1\equiv0[2]$ (A is poor at $t$)

\begin{gather*}
    \frac{\partial \mathcal{L}}{\partial c_t^R}=0\Leftrightarrow u'(c_t^R)=\lambda_t^R \\
    \frac{\partial \mathcal{L}}{\partial a_{t+1}^R}=0\Leftrightarrow \beta^t\lambda_t^R=\beta^{t+1}\lambda_{t+1}^P(1+r_{t+1})\Leftrightarrow \lambda_t^R=\beta\lambda_{t+1}^P(1+r_{t+1})
\end{gather*}

Which is the Euler equation when the agent is rich.\\

The same reasoning can be applied to agent 2 by changing the indicator in the Lagrangian. Indeed, we define the indicator $\mathbf{1}\{t\equiv0[2]\}$ which is equal to 0 when $t$ is even and 1 otherwise (i.e. when B is rich). Similarly, $\mathbf{1}\{t\equiv1[2]\}$ is equal to 1 when $t$ is even and 0 otherwise (i.e. when B is poor). This leads to the same results.\\

At the steady state:

\begin{gather*}
    u'(c^P)=\lambda^P\quad(1)\\
    \lambda^R=\beta\lambda^P(1+r)\quad(2)\\
    u'(c^R)=\lambda^R\quad(3)\\
    \lambda^P=\beta\lambda^R(1+r)+\mu^P \quad(4)
\end{gather*}

Combining (1) and (2), we get:
\begin{gather*}
    u'(c^R)=\beta u'(c^P)(1+r)\Leftrightarrow\frac{u'(c^R)}{u'(c^P)}=\beta (1+r)
\end{gather*}
Then, by complementary slackness $\mu^P>0$, this implies by (4):
\begin{gather*}
   \lambda^P>\beta\lambda^R(1+r)
\end{gather*}
By (3) and (1):
\begin{gather*}
    u'(c^P)>\beta u'(c^R)(1+r)\\
    \Leftrightarrow \frac{ u'(c^P)}{u'(c^R)} >\beta (1+r)
\end{gather*}
Because the marginal utility is positive.



\subsection{i}

The budget constraint of the households should bind as the utility is strictly increasing in consumption. They should use their endowment by either consuming tomorrow or saving for future consumption.\\

The budget constraint of the household is:

\begin{gather*}
    c_t^i+a^i_{t+1}= e_t^i+(1+r_t)a_t^i
\end{gather*}

When they are rich it becomes:

\begin{gather*}
    c_t^R+a^R_{t+1}= 1
\end{gather*}

Indeed, they cannot borrow when they are poor: $(1+r_t)a_t^P=0$.\\

At the steady state:

\begin{gather*}
    c^R+a^R=1
\end{gather*}

When they are poor, the households consume all their wealth $q_t^P=0+a_t^R(1+r_t)$:

\begin{gather*}
    c_t^P=a^R_t(1+r_t)
\end{gather*}

At the steady state:

\begin{gather*}
    c^P=a^R(1+r)
\end{gather*}

When they are rich, households save by lending to the entrepreneur at a net interest rate $r_t>0$. They can only save when they are rich. These savings are used by the entrepreneur to buy capital.  At the steady state, the amount that the households save plus the benefit of savings ($r\times a^R$) is equal to the amount borrowed by the entrepreneur to buy capital at the extra cost of borrowing  ($r \times a^R$). At the limit, the capital bought by the entrepreneur is equal to the savings of the households when they are rich. The cost of borrowing and the benefit from savings do not appear because their sum is 0 at the limit. Moreover, capital fully depreciates. From the law of motion of capital: $i_t=k_{t+1}$, as such the entrepreneurs cannot reuse past capital and should always borrow from households to buy new capital.

\begin{gather*}
    k=a^R
\end{gather*}

\subsection{j}

We assume: $ u(c)=\ln(c)$ and $ u'(c)=\frac{1}{c}$
We have the three equations:

\begin{gather*}
    \frac{u'(c^R)}{u'(c^P)}=\beta (1+r)\quad (1)\\
    c^R+a^R=1\quad (2)\\
    c^P=a^R(1+r)\quad (3)\\
    a^R=k\quad (4)
\end{gather*}

From the assumption about the utility function and (1):

\begin{gather*}
    \frac{\frac{1}{c^R}}{\frac{1}{c^P}}=\beta (1+r)\Leftrightarrow\frac{c^P}{c^R}=\beta (1+r)
\end{gather*}

From (3):

\begin{gather*}
    \frac{a^R}{c^R}(1+r)=\beta (1+r)\Leftrightarrow \frac{a^R}{c^R}=\beta\\
\end{gather*}

From (2):

\begin{gather*}
    \frac{a^R}{1-a^R}=\beta
\end{gather*}

From (4):

\begin{gather*}
    \frac{k}{1-k}=\beta\quad(5)
\end{gather*}

If $k=\frac{\beta}{1+\beta}$, we can show that (5) holds:

\begin{gather*}
    \frac{\frac{\beta}{1+\beta}}{1-\frac{\beta}{1+\beta}}=\beta\Leftrightarrow \frac{\frac{\beta}{1+\beta}}{\frac{1+\beta-\beta}{1+\beta}}\Leftrightarrow\frac{\frac{\beta}{1+\beta}}{\frac{1}{1+\beta}}\\
    \Leftrightarrow \beta=\beta
\end{gather*}

As such, we proved that $k=\frac{\beta}{1+\beta}$

\subsection{k}

From a: $\alpha k_t^{\alpha-1}=R_t$\\

At the steady state:

\begin{gather*}
    \alpha k^{\alpha-1}=1+r\\
    \Rightarrow \alpha \left ( \frac{\beta}{1+\beta} \right ) ^{\alpha-1}-1=r\quad\mbox{from j)}
\end{gather*}

The result is different from $1+r = \frac{1}{\beta}$, in fact it is a lower value. \\
Indeed, from question h

\begin{gather*}
    \frac{u'(c)^P}{u'(c)^R}>\beta (1+r)\quad (1)\\
    u'(c)^R=\beta u'(c)^P(1+r)\quad (2)\\
    \Leftrightarrow \frac{u'(c)^P}{u'(c)^R} \times \frac{u'(c)^R}{u'(c)^P}>(\beta (1+r))^2\\
    \Leftrightarrow 1>(\beta (1+r))^2\Leftrightarrow 1>\beta (1+r)
\end{gather*}
because $\beta (1+r)>0$. From (2), consumption, when agents are rich, will be lower than consumption when agents are poor. Indeed, the marginal gains of saving and reducing consumption when they are rich will be higher than the marginal cost of reducing saving. This suggests, that in the case of credit-constrained households, the income effect dominates. Agents hold precautionary savings even if the interest rate is lower than in the previous case. \\

Moreover, $1+r_t$ is the marginal productivity of capital which is decreasing in $k_t$. When the supply of assets increased, the entrepreneurs buy more capital. The marginal productivity of capital is decreasing, which implies a lower $r_t$ compared to the non-credit-constrained case.
\begin{gather*}
    \alpha \left ( \frac{\beta}{1+\beta} \right ) ^{\alpha-1}-\frac{1}{\beta} < 0
\end{gather*}

Indeed, we can graphically show (see Figure 1) that for the parameter values $\beta \in (0,1)$ and $\alpha \in (0,1)$, the interest rate is lower in the credit constraint case: \\


\color{black}

\newpage

\subsection{l}

Since the long-term interest rate is:

\begin{gather*}
    \Rightarrow \alpha \left ( \frac{\beta}{1-\beta} \right ) ^{1-\alpha}-1=r
\end{gather*}

it implies that there would be no change in the long-term interest rate, since the (taxed) endowment does not enter this expression. Moreover, since we do not tax savings or consumption specifically, the opportunity cost of savings remains the same in the long-run.


%\color{red} 

%Yes, it would be different. The reason being that if the endowment that the rich agent receives is now $1-T$ as opposed to its non-taxed value, the rich agent will have less to save and consume for the period when they become poor. \\

%Since $\hat{c}^R + \hat{a}^R = 1-T$ now, it implies that:

%\begin{gather*}
%    \hat{k} = \hat{a}^R \\
%    \hat{k} = 1-T-\hat{c}^R
%\end{gather*}

%Then, $\hat{k} < k$, which in turn implies that:

%\begin{gather*}
%    \alpha \hat{k_t}^{\alpha-1} = \hat{R_t} \\
%    \alpha (1-T-\hat{c}^R)^{\alpha-1} = \hat{R_t} \\
%    \implies \hat{R_t} \lesseqgtr R_t
%\end{gather*}

%As such, the interest rate change would be ambiguous with respect to taxation.

%\color{black}

\subsection{m}

Although the two interest rates are different for the case when the poor agent is credit constrained and is not credit constrained, the lump sum tax $T$ does not change the long-term interest rate in either case. This means that while the tax does not change the long-term interest rate, the credit constraint does. That's because the credit constraint changes the trade-off between present and future consumption because households can no longer freely save when they are poor and must hold precautionnary savings when they are rich. Indeed, we had that:

\begin{gather*}
    \frac{u'(c^R_t)}{u'(c^P_t)} = \beta(1+r) \\
    \frac{u'(c^P_t)}{u'(c^R_t)} > \beta(1+r).
\end{gather*}

which implies the relationship explicated above.
On the contrary, without the credit constraint, the trade off between present and future consumption does not change when the agent is poor:
\begin{gather*}
    \frac{u'(c^R_t)}{u'(c^P_t)} = \beta(1+r) \\
    \frac{u'(c^P_t)}{u'(c^R_t)} = \beta(1+r).
\end{gather*}
Agents are able to perfectly smooth their consumption over time while it is not the case when they are credit constrained. As such, the opportunity cost of saving should be different when they can freely save and when they cannot.


%Given that in a complete market setting the Ricardian equivalence holds, we expect that the introduction of a tax does not induce any changes with regard to the interest rate. That is because with Ricardian equivalence, the timing of a lump sum tax does not change consumption decisions, as households fully internalize the government's budget constraint. Meanwhile, it would change the interest rate with incomplete markets because the Ricardian equivalence no longer holds. The introduction of a tax changes consumption patterns, which in turn change saving patterns, leading to a different interest rate.\\

%\color{green}
%Vérifier les $\lambda^i$\\
%compléter justification avec doc supplémentaire donné today\\
%rajouter la dérivée par rapport à $lambda$ dans FOCs \\
%Vérifier les conditions initiales: what is given ?\\
%les check pour les endowments une fois que la consumption est trouvée
%\color{black}

\end{document}